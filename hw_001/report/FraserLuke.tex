\documentclass[letter]{IEEEtran}

\usepackage{amsmath}
\usepackage{amssymb}
\usepackage{graphicx}
\usepackage{algorithmic}
\usepackage{algorithm}

\title{Voltage Estimation}
\date{\today}
\begin{document}

\author{Luke Fraser}
\maketitle
\begin{abstract}
In this assignment we investigate the use of a kalman filter to estimate a variable. In the specifics of the problem we are estimating the true voltage of a noisy sensor reading. This simulates a more real-world application of the kalman filter and its use. The dimensionality of the sensor readings is $x \in \mathbb{R}^1$. This defines the space of our estimation as it is a single variable one dimensional space.
\end{abstract}

\section{Introduction}
The Kalman filter is an extremely useful tool for estimating noisy variables in a system. Its uses stretch far beyond uses in drone state estimation. It can be used within an IMU, as a tracker for computer vision, reading temperature readings, etc. Kalman filter uses extent into many domains. As a basic kalman filter is a linear estimator it is very fast and simple to use and implement making it perfect for different applications. Although the Kalman filter is a very capable estimator its simplicity limits is capabilities.

The Kalman filter due to its assumption of linearity it is only meant handle estimation of variable that abide by this assumption. In the case of state estimation of a quad-rotor pose in $\mathbb{R}^3$ the kalman filter does not perform as well. This is due to the non-linearity of the state of the drone. To fix this problem the extended kalman filter (EKF) was created to handle the non-linear nature of real-world state estimation problems.

Today EKF's are used to solve many complex estimation problems in real-time. A more simple alternative to drones is the estimation of a car. As the degrees of freedom are smaller EKF performance on cars has been well tuned and can be used to help produce self-driving cars.

\subsection{Kalman Filter}
The Kalman filter can be broken down into two stages:
\begin{itemize}
\item Prediction Step
\item Update Step
\end{itemize}
\section{Assignment}

\end{document}